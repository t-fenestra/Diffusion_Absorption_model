\documentclass[10pt, oneside]{article}
\usepackage{times}
\usepackage{graphicx}
\usepackage{lineno}
\usepackage{multirow}
\usepackage{hyperref}
\usepackage{amssymb}
\usepackage{amsfonts}
\usepackage{amsmath}
\usepackage{color}
\usepackage{natbib}
\usepackage{layout}
\usepackage{mathtools}
\usepackage{xcolor}
\usepackage{enumerate}
\usepackage{authblk}
\usepackage[top=2.5cm, bottom=2.5cm, inner=4cm, outer=2cm]{geometry}
\usepackage{caption}
\usepackage{subcaption}
\usepackage{todonotes}
\definecolor{darkblue}{rgb}{0,0,0.75}
\definecolor{darkred}{rgb}{0.75,0,0}
\definecolor{darkgreen}{rgb}{0,0.5,0}


\newcommand{\YP}[1]{\textbf{\textcolor{blue}{{YP: [}#1{]}}}}
\newcommand{\here}{\textbf{\textcolor{red}{ I am here!!! }}}

\renewcommand{\baselinestretch}{1.2}
\newcommand{\bbar}[1]{\overline{#1}}

\newcommand{\season}{\mathcal{S}}			%season
\newcommand{\env}{\mathcal{D}}			%environment
\newcommand{\LC}{\mathbf{q}}			%life cycle

%\linespread{1.6}
\setcounter{secnumdepth}{6}

%%%%%%%%%%%%%%%%%%%%%%%%%%%%%%%%%%%%%%%%%%%%%%%%%%%%%%%%%%%%
%%% ARTICLE SETUP
%%%%%%%%%%%%%%%%%%%%%%%%%%%%%%%%%%%%%%%%%%%%%%%%%%%%%%%%%%%%

\title {Kinetic model of biofilm formation}

%\author[1]{Tatyana Pichugina}
%\author[2]{Yuriy Pichugin}
%\affil[1]{Max Planck Institute for Evolutionary Biology, August-Thienemann-Str. 2, 24306 Pl\"on, Germany}


%\corr{pichugina@evolbio.mpg.de}{YP}


%%%%%%%%%%%%%%%%%%%%%%%%%%%%%%%%%%%%%%%%%%%%%%%%%%%%%%%%%%%%
%%% ARTICLE START
%%%%%%%%%%%%%%%%%%%%%%%%%%%%%%%%%%%%%%%%%%%%%%%%%%%%%%%%%%%%

\bibliographystyle{plainnat}

\begin{document}
\linenumbers

\maketitle

\tableofcontents
\clearpage
%
%\begin{abstract}
%Amazing abstract
%\end{abstract}


\part{Kinetic model of biofilm formation}\section{Introduction}

Consider a static liquid media with air-liquid interface (or glass-liquid).
Media is populated by the cells of type S (swimming), ALI is populated by cells of type A (attached).
The population of S-type cells is characterized by their density $S(x,t)$ changing with depth ($x$) and time ($t$).
The depth changes from $0$ (ALI) to $L$ (the bottom).
The population of A-type cells is characterized by its number at ALI $A(t)$ changing with time.
S-type cells can grow with the rate $w_S$, diffuse with coefficient $D$, and attach to the ALI at the border with coefficient $\kappa$ becoming cells of A-type.
A-type cells just grow with the rate $w_A$.

\section{Stochastic process to descibe reactive boundary }
This simple reactive-diffusion model can be described using deterministic partial-differential equation or stochastic simulation algorithms. These both approaches provide same description of the reactive-diffusion process far from the reactive boundary at the interface, but the behavior close to the reactive boundary is dependent on selected stochastic process (Erban and Chapman 2007). \par
Here we consider a stochastic simulation algorithm on the lattice size $[0,L]$ (Erban and Chapman 2007). We consider a system with $N$ particles of the planktonic type $S$ on the lattice lattice grids spaced $h$ distance apart. We choose the interval step $\Delta t$ so it satisfies stability condition ($ 2D\Delta t\ll h^{2}$) . The position of $i$-th molecule at the time $t$ is equal to $x_i (t)$. The position $x_i (t+\Delta t)$  is computed as follows.

\begin{equation}
x_i(t+\Delta t)=\begin{cases}
x_i(t), &\text{probability $=1-\frac{2D\Delta t}{h^{2}} $}\\
x_i(t)+h,  &\text{probability $=\frac{2 D \Delta t}{h^{2}} $}\\
x_i(t)-h,  &\text{probability $=\frac{2 D \Delta t}{h^{2}} $}\\
\end{cases}	
\end{equation}

If molecule hits the boundary $x=0$, it is adsorbed with probability $P_1h$ or reflected otherwise. Here $ P_1$ is non negative constant.

Let the probability finding a cell in the bulk satisfies.

\begin{equation}
p_k(t+\Delta t)=(1-\dfrac{2D\Delta t }{h^{2}})\cdot p_k(t)+
\dfrac{2D\Delta t }{h^{2}} \cdot (p_{k+1}(t) + p_{k-1}(t))
\end{equation}

Passing the limit $\Delta t \rightarrow 0,  h\rightarrow 0$ equation (2) 
we obtain diffusion equation.
\begin{equation}
\frac{p_k(t+\Delta t)-p_k(t)}{\Delta t}=
D \frac{p_k(t)+p_{k-1}(t)-2p_k(t)}{h^{2}}
\end{equation}

\begin{equation}
\frac{\partial n(x,t)}{ \partial t}=
D \cdot \bigtriangledown^2 n(x,t)
\end{equation}

Where $n(x,t)$ is a concentration of particles in space $x$ and time $t$.



The boundary condition can be incorporated into equation as 
\begin{equation}
p_1(t+\Delta t)=(1-\dfrac{2D\Delta t }{h^{2}})\cdot p_1(t)+
\dfrac{D\Delta t }{h^{2}} \cdot (p_{2}(t) +(1-P_1h) p_{1}(t))
\end{equation}

This equation can be rewritten as
\begin{equation}
\sqrt{\Delta t} \cdot \frac{p_1(t+\Delta t)-p_1(t)}{\Delta t}=
\frac{D \cdot \sqrt{\Delta t}}{h}  \cdot
\left (\frac{p_2(t)-p_1(t)}{h}-P_1p_1(t) \right )
\end{equation} 

Passing the limit $\Delta t \rightarrow 0,  h\rightarrow 0$  such that $ \sqrt{\Delta t}/h$ is kept constant we obtain boundary condition as
\begin{equation}
D\frac{p_2(t)-p_1(t)}{h}=P_1D \cdot p_1(t) 
\end{equation}

\begin{equation}
D\cdot \frac{\partial n}{ \partial x}(x=0,t)=
P_1D  \cdot n(x=0,t)
\end{equation}

Note that $P_1$ is a given non-negative constant and  $P_1h$ is probability of adsorbtion, where $h$ is a grid size.


\section{Kinetic equations}

The dynamics of S-type is given by the equation
%
\begin{linenomath}
	\begin{align} 
	\label{eq:s-dyn}
	\frac{\partial}{\partial t} S(x,t)= w_SS(x,t) + D \frac{\partial^2}{\partial x^2} S(x,t)
	\end{align}
\end{linenomath}
%
with boundary conditions
%
\begin{linenomath}
	\begin{align} 
	\label{eq:s-boundary}
	\begin{cases}
	D \frac{\partial}{\partial x} S(0,t) = P_1D \cdot S(0,t),  &\text{ partially adsorbing boader}\\\\ 
	\frac{\partial}{\partial x} S(L,t) = 0, &\text{ reflective boader}\\
	\end{cases}
	\end{align}
\end{linenomath}
%
where $w_S$ is a grownth rate of the S-type, $D$ is a diffusion coefficient, $P_1$ is a non-negative constant.
At $P_1 = 1$,  the boundary condition becomes absorbing $S(0,t) = 0$.
At $P_1 = 0$, the boundary condition becomes reflective $\frac{\partial}{\partial x} S(0,t) = 0$.
\newline
\newline
The growth dynamics of A-type is given by the equation
%
\begin{linenomath}
	\begin{align} 
	\label{eq:a-dyn}
	\frac{\partial}{\partial t} A = w_A A + P_1D \cdot S(0,t).
	\end{align}
\end{linenomath}
%
Where $w_A$ is a growth rate of A-type, term $P_1D \cdot S(0,t)$ is transition flux from S-type to A-type.

\section{Analytical solution}
We look for solution of the form $S(x,t)=X(x)T(t)\cdot e^{w_St}$. By substitution $S(x,t)$ into the PDE Eq.~\eqref{eq:s-dyn} and solving for X(x)T(t) we get.

\begin{linenomath}
	\begin{align} 
	\frac{T'}{DT}=\frac{X''}{X}=-\lambda^2
	\end{align}
\end{linenomath}

As a result we get
\begin{linenomath}
	\begin{align} 
	\label{eq:a-solution}
	S(x,t)=e^{w_St-\lambda^2Dt}\left(Bsin(\lambda x)+Ccos(\lambda x )\right)
	\end{align}
\end{linenomath}
Where $A,B$ and $ \lambda$ are arbitrary.\\

The next step is to find a certain subset of solutions that satisfy boundary conditions Eq.~\eqref{eq:s-boundary}. To do this we substitute ~\eqref{eq:a-solution} to the ~\eqref{eq:s-boundary}.

\begin{linenomath}
	\begin{align} 
	\begin{cases}
	D\lambda B=P_1DC \\
	B\lambda cos(\lambda L)-C \lambda sin(\lambda L)=0 \\
	\end{cases}
	\end{align}
\end{linenomath}

Last equations give us the condition on $\lambda$.
\begin{linenomath}
	\begin{align} 
    tan(\lambda L)=\dfrac{P_1}{\lambda}
	\end{align}
\end{linenomath}

To find $\lambda$ we must find the intersection of the curves $tan(\lambda L)$ and $P_1/\lambda$.
\begin{figure}[h!]
	\includegraphics[width=\linewidth]{tan_lambda.png}
	\caption{$\lambda_1, \lambda_2, \lambda_3 ...$ are the intersection of the curves $tan(\lambda L)$ and $P_1 L/\lambda L$}
	%\label{fig:boat1}
\end{figure}

 These values $\lambda_1, \lambda_2, \lambda_3 ...$ can be computed numerically and called eigenvalues of the boundary-value problem. In other words, they are values of $\lambda$ for which nonzero solution exsist.
 \\
 So we have found an infite number of fundamental solutions such as 
 \begin{linenomath}
 	\begin{align} 
 	\label{eq:a-fund-solution}
 	S_n=e^{w_St-\lambda^2Dt}\left(B_nsin(\lambda_n x)+C_ncos(\lambda_n x )\right)
 	\end{align}
 \end{linenomath}
Each of this solution satisfies PDE and boundary condition.
The analytical solution for $S(x,t)$ is
\begin{linenomath}
	\begin{align} 
	S(x,t)=\sum_{n=1}^{\infty} e^{w_St-\lambda_n^2Dt}\left(B_nsin(\lambda_n x)+C_ncos(\lambda_n x )\right)
	\end{align}
\end{linenomath}

We pick $B_n$, $C_n$ to satisfy the initial condition.$S(x,0)=f(x)$
\begin{linenomath}
	\begin{align} 
	f(x)=\sum_{n=1}^{\infty}\left(B_nsin(\lambda_n x)+C_ncos(\lambda_n x )\right)
	\end{align}
\end{linenomath}

\begin{linenomath}
	\begin{align} 
	\label{eq:a-dyn}
	\frac{\partial}{\partial t} A = w_A A + P_1D \cdot\sum_{n=1}^{\infty} e^{w_St-\lambda_n^2Dt}C_n .
	\end{align}
\end{linenomath}
 
\section{Stationary solution}
\subsection{S-type}

The fundamental solution  Eq.~\eqref{eq:a-fund-solution} has a decay term $ e^{-\lambda_n^2Dt}$  The smallest value of $\lambda_n$ - $\lambda_1$ determine behaviour of the system at the stationary regime.
To get $\lambda_1$ the Eq.~\eqref{eq:a-lambda1} equation should be solved numerically. 

\begin{linenomath}
	\begin{align} 
	\label{eq:a-lambda1}
	 tan(x)=\dfrac{P_1L}{x}\\
	\end{align}
	\end{linenomath}

where $x=\lambda L$\\
\\
However, some approximation can be done.  If $P_1L>>1$ then we can use the fact that $tan(\pi/2)=\infty$

\begin{linenomath}
	\begin{align} 
	\lambda = \frac{\pi}{2L} 
	\end{align}
\end{linenomath}	

\begin{linenomath}
	\begin{align} 
	S_1=e^{w_St-\frac{\pi^2}{4L^2}Dt}\left(B_1sin(\lambda x)+C_1cos(\lambda x )\right)
	\end{align}
\end{linenomath}
Note if $D>\frac{4L^2w_S}{\pi^2}$ S-type fades away.
	
 If $P_1L<<1$ then we can use the fact that $tan(x)\approx x$
\begin{linenomath}
	\begin{align} 
	\lambda = \sqrt{\frac{P_1}{L}}
\end{align}
\end{linenomath}

\begin{linenomath}
	\begin{align} 
	S_1=e^{w_St-\frac{P_1}{L}Dt}\left(B_1sin(\lambda x)+C_1cos(\lambda x )\right)
	\end{align}
\end{linenomath}


Note if $P_1>\frac{w_SL}{D}$ S-type fades away.



\subsection{A-type}
Stationary solution for A-type.

\begin{linenomath}
	\begin{align} 
	\frac{\partial}{\partial t} A = w_A A + P_1D \cdot\sum_{n=1}^{\infty} e^{w_St-\lambda_n^2Dt}C_n .
	\end{align}
\end{linenomath}

Subsitute  $\lambda = \lambda1$ and $x=0$
\begin{linenomath}
	\begin{align} 
	\frac{\partial}{\partial t} A = w_A A + P_1D e^{w_St-\lambda_1^{2}Dt} \cdot C .
	\end{align}
\end{linenomath}

Where $C$ depends on the initial conditions.\\
We look for solution in a form $A(t)=f(t)e^{\lambda_St}$ where $\lambda_S=w_S-\lambda_1^{2}D$.
\begin{linenomath}
	\begin{align} 
	f'(t)+(\lambda_S-w_A)f(t)-k = 0.
	\end{align}
\end{linenomath}
where $k=P_1DC$

\begin{linenomath}
	\begin{align} 
	f(t)=De^{-(\lambda_S-w_A)t}+\frac{k}{(\lambda_S-w_A)}
	\end{align}
\end{linenomath}

Taking into account $f(0)=0$, $D=-\frac{k}{\lambda_S-w_A}$

\begin{linenomath}
\begin{align} 
f(t)=\frac{k}{(\lambda_S-w_A)}\left( 1-e^{-(\lambda_S-w_A)t}\right)
\end{align}
\end{linenomath}

As a result
\begin{linenomath}
	\begin{align} 
	A(t)=\frac{k}{(\lambda_S-w_A)}\left( 1-e^{-(\lambda_S-w_A)t}\right)\cdot e^{\lambda_st}
	\end{align}
\end{linenomath}


\begin{linenomath}
	\begin{align} 
	A(t)=\frac{P_1DC}{(w_A-\lambda_S)}\cdot \big(e^{w_At}-e^{\lambda_st}\big)
	\end{align}
\end{linenomath}

Finally, by substitution $\lambda_S=w_S-\lambda_1^{2}D$

\begin{linenomath}
	\begin{align} 
	A(t)=\frac{P_1DC}{(w_A-w_S+\lambda_1^{2}D)}\cdot \big(e^{w_At}-e^{w_St-\lambda_1^{2}Dt}\big)
	\end{align}
\end{linenomath}

The number of A-cells grows by means of A-growth and S-type  transition.
Then, for each of two regimes, this value approaches.
\begin{enumerate}
	\item In the growth-dominated regime $w_A >> (w_S-\lambda_1^{2}D)$, $A(t)\sim \frac{P_1DС}{w_A}\cdot^{w_A t}$ 
	\item In the transition-dominated regime $w_A << (w_S-\lambda_1^{2}D)$, $A(t) \sim \frac{P_1DC}{w_S - \lambda_1^{2}Dt} \cdot e^{w_St-\lambda_1^{2}Dt}$.
\end{enumerate}






\section{Numerical solution}
To solve kinetic equations numerucally we calculated $S(x,t)$ on a lattice grid spacing in time $\Delta t$ and space  $h$ distance apart. We choose the interval step $\Delta t$ so it satisfies stability condition ($ 2D\Delta t\ll h^{2}$) . So,the  $S_k$ is number of S-cell  at the grid point  $k$ and $A$ is number of cells at the adsorbing boader.
\subsection{In the bulk}
\begin{linenomath}
	\begin{align} 
	\frac{\partial}{\partial t} S(x,t)= w_SS(x,t) + D \frac{\partial^2}{\partial x^2} S(x,t)
	\end{align}
\end{linenomath}

Can be rewritten as
\begin{linenomath}
	\begin{align} 
	S_k(t+\Delta t)=\left(1-\dfrac{2D \Delta t}{h^2}\right) \cdot S_k(t)+
	\dfrac{D \Delta t}{h^2}\cdot \Big(S_k(t)+ S_{k-1}(t)\Big)+
	w_S \Delta t\cdot S_k(t)
	\end{align}
\end{linenomath}

 

\subsection{At the  adsorbing boundary $x=0$}
\begin{linenomath}
	\begin{align}
	\begin{cases} 
	S_1(t+\Delta t)=\left(1-\dfrac{2D\Delta t}{h^2}\right) \cdot S_1(t)+
	\dfrac{D\Delta t}{h^2}\cdot \Big(S_2(t)+(1-P_1h)\cdot S_1(t)\Big)+
	w_S \Delta t\cdot S_1(t)\\
	A(t+\Delta t)=A(t)+w_A \Delta t \cdot A(t)+\dfrac{2D\Delta t}{h^2} \cdot P_1h\cdot S_1(t)\\
	\end{cases}
	\end{align}
\end{linenomath}

\subsection{At the  reflective boundary $x=L$}
\begin{linenomath}
	\begin{align} 
	S_L(t+\Delta t)=\left(1-\dfrac{2D\Delta t}{h^2}\right) \cdot S_L(t)+
	\dfrac{D\Delta t}{h^2}\cdot S_{L-1}(t)+
	w_S \Delta t\cdot S_L(t)
	\end{align}
\end{linenomath}


\section{Model with revertants}
\begin{linenomath}
	\begin{align} 
	\label{eq:s-revertant}
	\begin{cases}
	\frac{\partial}{\partial t} S(x,t)= w_SS(x,t) + D \frac{\partial^2}{\partial x^2} S(x,t)+w_{AS} A \delta(x) \\
	\frac{\partial}{\partial t} A = w_A A + P_1D \cdot S(0,t)-w_{AS} A\\	
	\end{cases}
	\end{align}
\end{linenomath}
%
with boundary conditions
%
\begin{linenomath}
	\begin{align} 
	\label{eq:s-boundary-revertants}
	\begin{cases}
	D \frac{\partial}{\partial x} S(0,t) = P_1D \cdot S(0,t),  &\text{ partially adsorbing boader}\\\\ 
	\frac{\partial}{\partial x} S(L,t) = 0, &\text{ reflective boader}\\
	\end{cases}
	\end{align}
\end{linenomath}
%
where $w_S$ is a grownth rate of the S-type, $D$ is a diffusion coefficient, $P_1$ is a non-negative constant.
At $P_1 = 1$,  the boundary condition becomes absorbing $S(0,t) = 0$.
At $P_1 = 0$, the boundary condition becomes reflective $\frac{\partial}{\partial x} S(0,t) = 0$. $w_A$ is a growth rate of A-type, term $P_1D \cdot S(0,t)$ is transition flux from S-type to A-type, $w_{AS}$ is a rate of transition from A-type to the S-type. $\delta(x)$ shows that revertants appear at the interface boarder.
\newline
\newline

\subsection{Bakward-propogation for numerical calculation}
A-type
\begin{linenomath}
	\begin{align} 
	\label{eq:s-a-type-revertants}
	A^n-A^{n-1}=w_AA^n\Delta t+P_1DS_1^{n}\Delta t-w_{AS}A^n \Delta t \\
	A^n=\dfrac{A^{n-1}+P_1DS_1^{n}\Delta t}{1+w_{AS}\Delta t-w_A\Delta t}
	\end{align}
\end{linenomath}

System of linear equations
\begin{linenomath}
	\begin{align} 
	\label{eq:s-bacwardS}
	\begin{cases}
	(1-w_A\Delta t+w_{AS}\Delta t) A^n-P_1\Delta x F S^n_1=A^{n-1}&\text{ a-type with reveltants}\\
	-w_{AS}\Delta t A^n+(1+2F-F(1-P_1h)-w_S\Delta t)S^n_1-FS^n_2=S^{n-1}_1 &\text{ partially adsorbing boarder}\\
	-FS^n_{i-1}+(1+2F-w_S\Delta t)S^n_{i}-FS^n_{i+1}=S^{n-1}_i,  &\text{ in the bulk}\\\\ 
	-FS^n_{N-1}+(1+F-w_S\Delta t)S^n_N=S^{n-1}_N &\text{ reflective boarder}\\
	\end{cases}
	\end{align}
\end{linenomath}


\label{key}

\end{document}
